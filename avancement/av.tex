% BASIC SETTINGS
\documentclass[a4paper,12pt]{article} 
\setlength {\marginparwidth }{2cm}
\usepackage[utf8]{inputenc} 
\usepackage{xcolor}
\usepackage{color}
% FORMAT SETTINGS
\usepackage[margin=20mm]{geometry}
%\setlength\parindent{0pt}
\usepackage{titling}
\usepackage{hyperref}
\hypersetup{colorlinks=true,linktoc=all,linkcolor=blue}
\usepackage{graphicx}
\usepackage{tcolorbox}
\usepackage{amsmath}
\usepackage{float}
\usepackage{lscape}
\usepackage{todonotes}
\usepackage[normalem]{ulem}


\usepackage{tikz}
\usetikzlibrary{shapes,automata,positioning,arrows.meta,petri,fit,trees,decorations.pathreplacing}



\usepackage{titlesec}

\setcounter{tocdepth}{4}
\setcounter{secnumdepth}{4}

\titleformat{\paragraph}
{\normalfont\normalsize\bfseries}{\theparagraph}{1em}{}
\titlespacing*{\paragraph}
{0pt}{3.25ex plus 1ex minus .2ex}{1.5ex plus .2ex}


% https://en.wikibooks.org/wiki/LaTeX/Source_Code_Listings
\usepackage{listings}
\definecolor{GrayCodeBlock}{RGB}{241,241,241}
% default
\lstset{ 
  backgroundcolor=\color{GrayCodeBlock},
}




\newcommand{\bs}{\vspace*{\baselineskip}}

\newcommand{\tdi}[1]{\todo[inline]{{#1}}{}}



\title{Stage de fin d'études\\ \vspace*{20mm} \scalebox{2}{Rapport d'avancement}\\ \vspace*{20mm} Analyse des mémoires caches pour le WCET par identification de chemins créant des défauts de cache}
\author{Elana Courtines}
\date{7 mars 2024}


\begin{document}

\maketitle


\section{Semaine 1 (8 mars 2024)}

(Re)prise en main de OTAWA :

\begin{itemize}
  \item installation du logiciel sur ma machine ; découverte de bugs dans le fichier d'installation du fait de versions trop récentes de mes "applications"
  \item avancement dans le tutoriel OTAWA - cur : 2.0
  \item lecture des sections spécifiées dans la thèse de Clément BALLABRIGA sur l'analyse WCET
  \item \sout{découverte de typst}
  \item une partie du temps de la semaine a été \textit{emprunté} par l'écriture du rapport du projet long
  \item apparemment il vaudrait mieux que je ne rentre pas dans l'IRIT :thinking: je n'ai aussi plus de bourse le temps que mon stage démarre 
\end{itemize}


\section{Semaine 2 (15 mars 2024)}


\begin{itemize}
  \item Tutoriel OTAWA la suite
  \item Fin de la lecture des articles
\end{itemize}



\section{Semaine 3 (22 mars 2024)}

\begin{itemize}
  \item Fin du tutoriel
  \item Mise un place d'un programme d'analyse de cfg
\end{itemize}

\section{Semaine 4 (29 mars 2024)}

\begin{itemize}
  \item première "release" d'un programme permettant de récolter tous les états du cache (fonctionnel) pour un programme donné
\end{itemize}

\section{Semaine 5 (5 avril 2024)}

\begin{itemize}
  \item Correction de bogues et améliorations
  \item ajout de politiques de remplacement de cache (LRU, FIFO et PLRU)
  \item ajout de commentaires (WIP)
  \item première campagne de tests
  \item programme réalisé en working list (5 avril)
\end{itemize}


TODO : refractorisation du code, séparation en plusieurs fichiers ?

\section{Semaine 6 (12 avril 2024)}

\begin{itemize}
  \item Correction de bogues et améliorations
  \item amélioration working list (n'était pas tout à fait correct)
  \item benchmarking complet pour détection de cas particuliers
  \item script de stats (WIP)
\end{itemize}


\section{Semaine 7 (19 avril 2024)}

\begin{itemize}
  \item ajout de commentaires (Done?)
  \item script de stats (Done?)
\end{itemize}

\end{document}