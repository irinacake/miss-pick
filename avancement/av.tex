% BASIC SETTINGS
\documentclass[a4paper,12pt]{article} 
\setlength {\marginparwidth }{2cm}
\usepackage[utf8]{inputenc} 
\usepackage{xcolor}
\usepackage{color}
% FORMAT SETTINGS
\usepackage[margin=20mm]{geometry}
%\setlength\parindent{0pt}
\usepackage{titling}
\usepackage{hyperref}
\hypersetup{colorlinks=true,linktoc=all,linkcolor=blue}
\usepackage{graphicx}
\usepackage{tcolorbox}
\usepackage{amsmath}
\usepackage{float}
\usepackage{lscape}
\usepackage{todonotes}
\usepackage[normalem]{ulem}


\usepackage{tikz}
\usetikzlibrary{shapes,automata,positioning,arrows.meta,petri,fit,trees,decorations.pathreplacing}



\usepackage{titlesec}

\setcounter{tocdepth}{4}
\setcounter{secnumdepth}{4}

\titleformat{\paragraph}
{\normalfont\normalsize\bfseries}{\theparagraph}{1em}{}
\titlespacing*{\paragraph}
{0pt}{3.25ex plus 1ex minus .2ex}{1.5ex plus .2ex}


% https://en.wikibooks.org/wiki/LaTeX/Source_Code_Listings
\usepackage{listings}
\definecolor{GrayCodeBlock}{RGB}{241,241,241}
% default
\lstset{ 
  backgroundcolor=\color{GrayCodeBlock},
}




\newcommand{\bs}{\vspace*{\baselineskip}}

\newcommand{\tdi}[1]{\todo[inline]{{#1}}{}}



\title{Stage de fin d'études\\ \vspace*{20mm} \scalebox{2}{Rapport d'avancement}\\ \vspace*{20mm} Analyse des mémoires caches pour le WCET par identification de chemins créant des défauts de cache}
\author{Elana Courtines}
\date{7 mars 2024}


\begin{document}

\maketitle

\textit{Date : vendredi de la semaine}

\section{Semaine 1 (8 mars 2024)}

(Re)prise en main de OTAWA :

\begin{itemize}
  \item installation du logiciel sur ma machine ; découverte de bugs dans le fichier d'installation du fait de versions trop récentes de mes "applications"
  \item avancement dans le tutoriel OTAWA - cur : 2.0
  \item lecture des sections spécifiées dans la thèse de Clément BALLABRIGA sur l'analyse WCET
  \item \sout{découverte de typst}
  \item une partie du temps de la semaine a été \textit{emprunté} par l'écriture du rapport du projet long
  \item apparemment il vaudrait mieux que je ne rentre pas dans l'IRIT :thinking: je n'ai aussi plus de bourse le temps que mon stage démarre 
\end{itemize}


\section{Semaine 2 (15 mars 2024)}


\begin{itemize}
  \item Tutoriel OTAWA la suite
  \item Fin de la lecture des articles
\end{itemize}



\section{Semaine 3 (22 mars 2024)}

\begin{itemize}
  \item Fin du tutoriel
  \item Mise un place d'un programme d'analyse de cfg
\end{itemize}

\section{Semaine 4 (29 mars 2024)}

\begin{itemize}
  \item première "release" d'un programme permettant de récolter tous les états du cache (fonctionnel) pour un programme donné
\end{itemize}

\section{Semaine 5 (5 avril 2024)}

\begin{itemize}
  \item Correction de bogues et améliorations
  \item ajout de politiques de remplacement de cache (LRU, FIFO et PLRU)
  \item ajout de commentaires (WIP)
  \item première campagne de tests
  \item programme réalisé en working list (5 avril)
\end{itemize}


TODO : refractorisation du code, séparation en plusieurs fichiers ?

\section{Semaine 6 (12 avril 2024)}

\begin{itemize}
  \item Correction de bogues et améliorations
  \item amélioration working list (n'était pas tout à fait correct)
  \item benchmarking complet pour détection de cas particuliers
  \item script de stats (WIP)
\end{itemize}


\section{Semaine 7 (19 avril 2024)}

\begin{itemize}
  \item ajout de commentaires
  \item script de stats
  \item préparation à une refractorisation
\end{itemize}
  

\section{Semaine 8 (26 avril 2024)}

\begin{itemize}
  \item refractorisation du code
  \item mise en place d'un processeur de code
  \item problèmes de mémoire, utilisation de instruments (MacOS)
\end{itemize}

\section{Semaine 9 (3 mai 2024)}

\begin{itemize}
  \item des jours fériés
  \item mon déménagement, je n'ai pas pu beaucoup travailler
\end{itemize}

\section{Semaine 10 (10 mai 2024)}

\begin{itemize}
  \item nouvelle refractorisation en reprenant la structure de 0
  \item découverte de bugs majeurs
\end{itemize}

\section{Semaine 11 (17 mai 2024)}

\begin{itemize}
  \item correction de bugs
  \item usage d'AVL trees
  \item simulation de pile (callstack)
  \item début de la projection par ensemble
\end{itemize}

\section{Semaine 12 (24 mai 2024)}

\begin{itemize}
  \item deboggage de la projection
  \item finalisation de la projection
  \item traduction de l'algorithme "process" vers le CFG projeté
  \item découverte du bug "bottleneck premature stoppage"...
\end{itemize}

\section{Semaine 13 (31 mai 2024)}

\begin{itemize}
  \item correction du bottleneck (plusieurs jours)
  \item ajout de simplifications des CFG projetés (WIP) :
  \begin{itemize}
    \item suppression de self-loop sur bloc inutile
    \item suppression de sous-cfg non impliqué
    \item suppression des bloc synth appelant des sous-cfg non impliqués
  \end{itemize}
  \textbf{Spécial :} Conférences OSPM Summit (jeudi toute la journée, vendredi matin)
\end{itemize}

\section{Semaine 14 (7 juin 2024)}

\begin{itemize}
  \item finalisation des simplifications de projection
  \item correction du bottleneck pour la version non projeté (dans l'optique de permettre des comparaisons de performance)
  \item début de la réflexion sur l'analyse de miss
  \item \textbf{Spécial :} Journée des doctorants (mardi, toute la journée)
  \item \textbf{Spécial :} Soutenance HDR (jeudi, matin)
\end{itemize}

\section{Semaine 15 (14 juin 2024)}

\begin{itemize}
  \item mise en place de wsl / WakeOnLan pour accéder à ma tour (option linux pour valgrind)
  \item progrès sur les leaks (il ne reste presque plus rien)
  \item corrections de bugs
  \item analyse de miss : établissement d'un squelette d'algorithme (à la main) pour récupérer "qui a éjecté qui"
  \item \textbf{Spécial :} Conférence - CAPITAL Workshop 2024 (vendredi, toute la journée)
\end{itemize}

\section{Semaine 16 (21 juin 2024)}

\begin{itemize}
  \item ~
\end{itemize}

\section{Semaine 17 (28 juin 2024)}

\begin{itemize}
  \item ~
\end{itemize}

\section{Semaine 18 (5 juillet 2024)}

\begin{itemize}
  \item ~
\end{itemize}

\section{Semaine 19 (12 juillet 2024)}

\begin{itemize}
  \item ~
\end{itemize}

\section{Semaine 20 (19 juillet 2024)}

\begin{itemize}
  \item ~
\end{itemize}

\section{Semaine 21 (26 juillet 2024)}

\begin{itemize}
  \item ~
\end{itemize}

\section{Semaine 22 (1 août 2024)}

\begin{itemize}
  \item ~
\end{itemize}

\section{Semaine 23 (8 août 2024)}

\begin{itemize}
  \item ~
\end{itemize}


\end{document}