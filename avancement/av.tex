% BASIC SETTINGS
\documentclass[a4paper,12pt]{article} 
\setlength {\marginparwidth }{2cm}
\usepackage[utf8]{inputenc} 
\usepackage{xcolor}
\usepackage{color}
% FORMAT SETTINGS
\usepackage[margin=20mm]{geometry}
%\setlength\parindent{0pt}
\usepackage{titling}
\usepackage{hyperref}
\hypersetup{colorlinks=true,linktoc=all,linkcolor=blue}
\usepackage{graphicx}
\usepackage{tcolorbox}
\usepackage{amsmath}
\usepackage{float}
\usepackage{lscape}
\usepackage{todonotes}
\usepackage[normalem]{ulem}


\usepackage{tikz}
\usetikzlibrary{shapes,automata,positioning,arrows.meta,petri,fit,trees,decorations.pathreplacing}



\usepackage{titlesec}

\setcounter{tocdepth}{4}
\setcounter{secnumdepth}{4}

\titleformat{\paragraph}
{\normalfont\normalsize\bfseries}{\theparagraph}{1em}{}
\titlespacing*{\paragraph}
{0pt}{3.25ex plus 1ex minus .2ex}{1.5ex plus .2ex}


% https://en.wikibooks.org/wiki/LaTeX/Source_Code_Listings
\usepackage{listings}
\definecolor{GrayCodeBlock}{RGB}{241,241,241}
% default
\lstset{ 
  backgroundcolor=\color{GrayCodeBlock},
}




\newcommand{\bs}{\vspace*{\baselineskip}}

\newcommand{\tdi}[1]{\todo[inline]{{#1}}{}}



\title{Stage de fin d'études\\ \vspace*{20mm} \scalebox{2}{Rapport d'avancement}\\ \vspace*{20mm} Analyse des mémoires caches pour le WCET par identification de chemins créant des défauts de cache}
\author{Elana Courtines}
\date{7 mars 2024}


\begin{document}

\maketitle


\section{Semaine 1}

(Re)prise en main de OTAWA :

\begin{itemize}
  \item installation du logiciel sur ma machine ; découverte de bugs dans le fichier d'installation du fait de versions trop récentes de mes "applications"
  \item avancement dans le tutoriel OTAWA - cur : 2.0
  \item lecture des sections spécifiées dans la thèse de Clément BALLABRIGA sur l'analyse WCET
  \item \sout{découverte de typst}
  \item une partie du temps de la semaine a été \textit{emprunté} par l'écriture du rapport du projet long
  \item apparemment il vaudrait mieux que je ne rentre pas dans l'IRIT :thinking: je n'ai aussi plus de bourse le temps que mon stage démarre 
\end{itemize}



\section{Semaine 1}

\begin{itemize}
  \item Tutoriel OTAWA la suite
\end{itemize}


\end{document}